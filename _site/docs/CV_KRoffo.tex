% LaTeX Curriculum Vitae Template
%
% Copyright (C) 2004-2009 Jason Blevins <jrblevin@sdf.lonestar.org>
% http://jblevins.org/projects/cv-template/
%
% You may use use this document as a template to create your own CV
% and you may redistribute the source code freely. No attribution is
% required in any resulting documents. I do ask that you please leave
% this notice and the above URL in the source code if you choose to
% redistribute this file.

\documentclass[8pt]{article}

\usepackage{hyperref}
\usepackage{geometry}

% Comment the following lines to use the default Computer Modern font
% instead of the Palatino font provided by the mathpazo package.
% Remove the 'osf' bit if you don't like the old style figures.
%\usepackage[T1]{fontenc}
%\usepackage[sc]{mathpazo}

% Set your name here
\def\name{Kenneth R. Roffo Jr.}

% Replace this with a link to your CV if you like, or set it empty
% (as in \def\footerlink{}) to remove the link in the footer:
\def\footerlink{}

% The following metadata will show up in the PDF properties
\hypersetup{
  colorlinks = true,
  urlcolor = blue,
  pdfauthor = {\name},
  pdfkeywords = {mathematics},
  pdftitle = {\name: Curriculum Vitae},
  pdfsubject = {Curriculum Vitae},
  pdfpagemode = UseNone
}

\geometry{
  body={6.5in, 8.5in},
  left=1.0in,
  top=1.25in
}

% Customize page headers
%\pagestyle{myheadings}
%\markright{\name}
%\thispagestyle{empty}


% Other possible font commands include:
% \ttfamily for teletype,
% \sffamily for sans serif,
% \bfseries for bold,
% \scshape for small caps,
% \normalsize, \large, \Large, \LARGE sizes.

% Don't indent paragraphs.
\setlength\parindent{0em}

% Make lists without bullets
\renewenvironment{itemize}{
  \begin{list}{}{
    \setlength{\leftmargin}{1.5em}
  }
}{
  \end{list}
}

\begin{document}

% Place name at left
{\huge \bf \name}

% Alternatively, print name centered and bold:
%\centerline{\huge \bf \name \\}

\large \href{mailto:kroffo@oswego.edu}{kroffo@oswego.edu}\\
\href{https://github.com/kroffo}{https://github.com/kroffo}\\

\vspace{0.15in}

20 Wallace Road \\
Phoenix, NY 13135 \\
(315)-214-1889 (Cell)


\section*{\textsc{Education}}

\begin{itemize}
  \item B.S., Physics, Mathematics, Computer Science \hfill anticipated 2017\\
  \textbf{SUNY Oswego - GPA 3.68}
  \item New York State Advanced Regents Diploma with Honors \hfill 2012\\
  \textbf{John C. Birdlebough High School - GPA 91}
  \end{itemize}	

\section*{Internships} See \href{https://www.youtube.com/watch?v=BiJOjhm_ov4}{this} interview about my internships and tutoring SUNY Oswego.
\begin{itemize}
\item \textbf{Software Engineer - NASA Jet Propulsion Laboratory}\\
  The Deep Space Network consists of multiple antennae on Earth which communicate with space craft beyond the moon. In order to improve this process, NASA software engineers are developing a new software to generate files read by the antennae, however they must check that the new software does not generate files with errors. My project at JPL was to develop a diff tool using node.js which would compare these files, and display differences, which the users could flag as unimportant differences, or more imporantly find defects in the products of their software.\\
  Mentor: \textbf{Mark Johnston} \hfill Summer 2015

%\item \textbf{Asteroseismologist - Max-Planck-Institut für Sonnensystemforschung (MPS)\\
%  Add description here
%  Advisor: \textbf{Saskia Hekker} \hfill Summer 2016

\end{itemize}


%\section*{\textsc{Conferences}}
%\begin{itemize}
%  \item MAA Seaway Section Meeting, Colgate University \hfill 2015 
%\end{itemize}
 
 
% \section*{\textsc{Service}}
% 	\begin{itemize}
%	\item Graduate Judicial Board, Wesleyan University. \hfill 2013-2014
%	%\end{itemize}



\section*{Research Experience}
\begin{itemize}

\item \textbf{An Asteroseismic Analysis of the Red Giant Branch Bump}\\
As an intern the Max-Planck Institute for Solar System Research in G{\"o}ttingen, Germany I studied how asterosesmic parameters were effected during the RGB bump. I used the MESA
stellar evolution code to generate tracks of models of stars with varying masses, then used ADIPLS to calculate the frequencies they would output as the stars passed through the
bump.\\
Advisors: \textbf{Saskia Hekker, Earl Bellinger, George Angelou} \hfill Summer 2016

\item \href{https://kroffo.github.io/docs/thesis.pdf}{\textbf{The Application of Abstract Algebra to Twisty Puzzles}}\\
Rubik's Cubes have fascinated mathematicians ever since they made their debut in the 1970s. Since then, many differently shaped and sized variants of the Rubik's Cube (called twisty puzzles) have become available. In this research I applied concepts I learned in Abstract Algebra to describe these fascinating puzzles. I also worked on a design for a \href{https://www.youtube.com/watch?v=17ll6TWm45M}{puzzle which I have created}, and 3D-printed thanks to SUNY Oswego's SCAC grant.\\
Advisors: \textbf{Bonita Graham, David Vampola} \hfill Fall 2014 - Present

\item \textbf{Fourier Decomposition Analysis of CSTAR RR Lyrae Variable Stars}\\
I began this research through a 6 week visit to India in summer 2014. My original, and now completed, goal was to determine the metallicities of several RR Lyrae variable stars.\\
Advisor: \textbf{Shashi Kanbur} \hfill Summer 2014 - Present

\end{itemize}

\section*{Teaching}
\begin{itemize}
\item \textbf{Math Club Tutoring} \hfill 2015-Present\\
Organized and participated in free Math Club tutoring sessions for Calculus students.
\item \textbf{Math and Sciences Tutor at SUNY Oswego} \hfill 2014-Present\\
Courses Tutored: Calculus 1, 2 and 3, Discrete Math, Physics 1 and 2, CS intro level
\item \textbf{HON 150 Seminar Leader at SUNY Oswego} \hfill Fall 2014\\
Prepared and presented weekly lectures for an introduction-to-college course. Created and Graded weekly writing assignments.
\end{itemize}

\section*{\textsc{Talks}}

\begin{itemize}
\item \emph{A New Cube}. \\
  MAA Seaway Section Meeting, SUNY Geneseeo \hfill 2016
\item \emph{The Invention of a Cube}.\\
  Quest, SUNY Oswego \hfill 2016
\item \href{http://kroffo.github.io/presentations/SeawaySpr15.pdf}{\emph{A Necessary Set of Turns to Solve a Rubik's Cube}}.\\
  MAA Seaway Section Meeting, Colgate University \hfill 2015
\item \emph{The Necessity and Sufficiency of 5 Face Turns to Solve a Rubik's Cube}.\\
  Quest, SUNY Oswego \hfill 2015
\item \emph{RR Lyrae Metallicities from CSTAR data}.\\
  Quest, SUNY Oswego \hfill 2015
\item \emph{Fourier Analysis of CSTAR RR Lyrae Variable Stars}.\\
  Rochestor Symposium for Physics Students, SUNY Oswego \hfill 2015
\item \href{http://kroffo.github.io/presentations/RRLyraeMetallicity2015.pdf}{\emph{Metallicity determination for RR Lyraes observed from CSTAR telescopes in Antarctica}}.\\
  SUNY Undergraduate Research Conference, SUNY Brockport \hfill 2015	
\item \emph{The Line Trick to Multiplying Numbers and Polynomials}.\\
  Math Club, SUNY Oswego \hfill 2015
  
\end{itemize}


\section*{\textsc{Honors}}

\begin{itemize}
\item Honors Program - SUNY Oswego\hfill 2012-Present
\item \href{http://oswegocountytoday.com/local-residents-receive-suny-oswego-scholarships/}{Presidential Scholarship for Academic Achievement} - SUNY Oswego  \hfill 2012-Present
\item Student Involvement Award - SUNY Oswego \hfill Spring 2015
\item Sigma Xi and Office for Research and Sponsored Programs Award for Excellence in Research Presentation - SUNY Oswego \hfill Spring 2015
\item \href{http://meritpages.com/Kroffo}{Dean's List} - SUNY Oswego \hfill Fall 2014 - Spring 2015
\item \href{http://meritpages.com/Kroffo}{President's List} - SUNY Oswego \hfill Fall 2012 - Fall 2013, Spring 2016
\item \href{http://www.oswegocountyweeklies.com/phoenix_register.php?details&story_id=5101&story_year=2012&story_month=5}{Youth of the Year} - John C. Birdlebough High School \hfill 2012
\item Presidential Community Service Award - Corporation for National and Community \\Service\hfill 2012
\item \href{http://www.oswegocountyweeklies.com/phoenix_register.php?details&story_id=5503&story_year=2012&story_month=6}{Senior Key in Mathematics} - John C. Birdlebough High School \hfill 2012
\item \href{http://www.oswegocountyweeklies.com/phoenix_register.php?details&story_id=6083&story_year=2012&story_month=8}{Eagle Scout} - Boy Scouts of America \hfill 2011
\end{itemize}

\section*{Membership}
\begin{itemize}
\item SUNY Oswego Physics Club \hfill 2014-Present
\item SUNY Oswego Astronomy Club - \emph{Treasurer} \hfill 2013-Present
\item SUNY Oswego Math Club - \emph{President} \hfill 2012-Present
\item Omicron Delta Kappa National Leadership Honor Society \hfill Inducted 2015
\item Phi Kappa Phi National Honor Society \hfill Inducted 2014
\item National Honor Society \hfill Inducted 2010
\item Tri-M Music National Honor Society \hfill Inducted 2010
\item John C. Birdlebough HS Student Council - \emph{President} \hfill 2010-2012
\item Boy Scouts of America - \emph{Quartermaster, Assistant Senior Patrol Leader}\hfill 1999-2012
\end{itemize}

\section*{Skills}
\begin{itemize}
\item Mac and Linux Proficient
\item Proficient in Bash, Python, Java, Fortran, {\LaTeX}, C/C++, Javascript, and HTML/CSS
\item \href{https://youtu.be/qPBg2xok04s}{Rubik's Cube Speed Solver}
\end{itemize}

\bigskip

% Footer
\begin{center}
  \begin{footnotesize}
    Last updated: \today \\
    \href{\footerlink}{\texttt{\footerlink}}
  \end{footnotesize}
\end{center}

\end{document}
