%%%%%%%%%%%%%%%%%%%%%%%%%%%%%%%%%%%%%%%%%
% Medium Length Professional CV
% LaTeX Template
% Version 2.0 (8/5/13)
%
% This template has been downloaded from:
% http://www.LaTeXTemplates.com
%
% Original author:
% Trey Hunner (http://www.treyhunner.com/)
%
% Modifications made by:
% Daniel Wysocki (dwysocki.github.io)
%
% Important note:
% This template requires the resume.cls file to be in the same directory as the
% .tex file. The resume.cls file provides the resume style used for structuring
% the document.
%
%%%%%%%%%%%%%%%%%%%%%%%%%%%%%%%%%%%%%%%%%

\documentclass{resume} % Use the custom resume.cls style

\usepackage[left=0.75in,top=0.6in,right=0.75in,bottom=0.6in]{geometry}
\usepackage{hyperref}
\hypersetup{
  hidelinks,
  colorlinks,
  urlcolor=blue
}
\usepackage{graphicx}

% dingbats fonts
\usepackage{bbding}

\newcommand*\wwwicon{\includegraphics[height=2ex]{globe.eps}}


\name{Kenny Roffo}
\info{
  \wwwicon\hspace{0.10cm}
  \href{https://kroffo.github.io/}
       {https://kroffo.github.io/}
  \\
  \faGithub\hspace{0.10cm}\href{https://github.com/kroffo/}{\texttt{kroffo}}
}

\begin{document}

\textbf{State University of New York at Oswego}
\\
B.S. Physics, Mathematics, Computer Science, Honors Program, 3.68
\hfill
expected \emph{May 2017}

\begin{rSection}{Research \& Internship Experience}{}

  \position{NASA Jet Propulsion Laboratory}
           {Pasadena, CA, USA}
           {September 2016 -- present}
           {Software Computing Systems Undergraduate Student IV}
           {
    \textbf{Advisors:}
    Diane Conner and Mark Johnston
    \\
    \textbf{Topic:}
    Development of a web-based tool for scheduling maintenance activities on
    antennae of the Deep Space Network.
  }

 \position{\href{http://www.mps.mpg.de/sage}{SAGE}, Max-Planck Institute for Solar System Research}
           {Goettingen, Germany}
           {Summer 2016}
           {Research Assistant}
           {
    \textbf{Advisors:}
    Saskia Hekker, George Angelou, Earl Bellinger, Shashi M. Kanbur
    \\
    \textbf{Topic:}
    An asteroseismic analysis of the red giant branch bump using MESA stellar
    evolution code and ADIPLS stellar pulsation code.
    \\
    \textbf{Grant:}
    SUNY Oswego Challenge grant
  }

  \position{NASA Jet Propulsion Laboratory}
           {Pasadena, CA, USA}
           {Summer 2015}
           {Summer Intern}
           {
    \textbf{Advisors:}
    Diane Conner and Mark Johnston
    \\
    \textbf{Topic:}
    Development of a web-based diff tool to greatly increase the speed at which
    software engineers at JPL could test new versions of a tool under
    development.
  }

  \position{Department of Physics \& Astrophysics, University of Delhi}
           {New Delhi, India}
           {Summer 2014}
           {Research Assistant}
           {
    \textbf{Advisors:}
    Shashi M. Kanbur, H. P. Singh
    \\
    \textbf{Topic:}
    Determination of metallicities of several RR Lyrae variable stars in the
    CSTAR data sets using Fourier Decomposition.
    \\
    \textbf{Grant:}
    SUNY Oswego Challenge grant
  }

\end{rSection}

\begin{rSection}{Awards}{}

  \textbf{Sigma Xi / ORSP Quest 2015 Award}, SUNY Oswego
  \hfill
  \emph{2015}
  
  \textbf{Presidential Scholarship for Academic Achievement}, SUNY Oswego
  \hfill
  \emph{2012-2016}

\end{rSection}

\begin{rSection}{Organizations}{}

  \textbf{Omicron Delta Kappa National Leadership Honor Society}
  \hfill
  \emph{Inducted 2015}
  
  \textbf{SUNY Oswego Math Club}, President
  \hfill
  \emph{2014 -- Present}
  
  \textbf{SUNY Oswego Astronomy Club}, Treasurer
  \hfill
  \emph{2013 -- Present}
  
  \textbf{Boy Scouts of America}, Eagle Scout
  \hfill
  \emph{1999 -- 2012}
  
\end{rSection}


\begin{rSection}{Technical Skills}{}

\begin{tabular}{ @{} >{\bfseries}l @{\hspace{6ex}} l }
Programming Languages &
Java, JavaScript, Python, Bash, C/C++
\\
Markup Languages &
\LaTeX, HTML, markdown
\\
Tools & Linux, Git, Emacs
\end{tabular}

\end{rSection}

\end{document}
