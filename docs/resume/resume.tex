%%%%%%%%%%%%%%%%%%%%%%%%%%%%%%%%%%%%%%%%%
% Medium Length Professional CV
% LaTeX Template
% Version 2.0 (8/5/13)
%
% This template has been downloaded from:
% http://www.LaTeXTemplates.com
%
% Original author:
% Trey Hunner (http://www.treyhunner.com/)
%
% Modifications made by:
% Daniel Wysocki (dwysocki.github.io)
%
% Important note:
% This template requires the resume.cls file to be in the same directory as the
% .tex file. The resume.cls file provides the resume style used for structuring
% the document.
%
%%%%%%%%%%%%%%%%%%%%%%%%%%%%%%%%%%%%%%%%%

\documentclass{resume} % Use the custom resume.cls style

\usepackage[left=0.5in,top=0.1in,right=0.5in,bottom=0.1in]{geometry}
\usepackage{hyperref}
\hypersetup{
  hidelinks,
  colorlinks,
  urlcolor=blue
}
\usepackage{graphicx}

% dingbats fonts
\usepackage{bbding}

\newcommand*\wwwicon{\includegraphics[height=2ex]{globe.eps}}
\newcommand*\mailicon{\includegraphics[height=2ex]{Letter.png}}

\name{Kenny Roffo}
\info{
  \mailicon\hspace{0.10cm}
  \href{mailto:kroffo@oswego.edu}{kroffojr@gmail.com}
  \wwwicon\hspace{0.10cm}
  \href{http://kennyroffo.com}
       {kennyroffo.com}
  \\
  \faGithub\hspace{0.10cm}\href{https://github.com/kroffo/}{\texttt{kroffo}}
}

\begin{document}

\textbf{State University of New York at Oswego}
\\
B.S. Physics, Mathematics, Computer Science, Honors Program, 3.66
\hfill
\emph{May 2017}

\begin{rSection}{Professional Experience}{}
  \position{NASA Jet Propulsion Laboratory}
           {Pasadena, CA, USA}
           {June 2017 -- Present}
           {Software Engineer}
           {
        \textbf{Notable Contributions:}\\
        - \href{http://www.oswegocountynewsnow.com/news/mars-landing-puts-phoenix-native-suny-oswego-grad-roffo-among/article_968bd574-f34c-11e8-8c5b-b7d557e9f682.html?fbclid=IwAR1iCW3CdQIRr2JIy1f96utXjzzidSJ6vA13jVgvkjKIU6PopaLHYkpbQrw}{InSight Mars Lander Development and Operation of Surface Modeling Tools}\\
        - \href{https://github.com/NASA-AMMOS/aerie}{Ongoing development of the Aerie tool suite for mission planning (Open Source)}
    }
\end{rSection}

\begin{rSection}{Research \& Internship Experience}{}

  \position{NASA Jet Propulsion Laboratory}
           {Pasadena, CA, USA}
           {September 2016 -- December 2016}
           {Software Computing Systems Undergraduate Student IV}
           {
    \textbf{Advisors:}
    Diane Conner and Mark Johnston
    \\
    \textbf{Topic:}
    Development of a web-based tool for scheduling maintenance activities for the DSN.
  }

 \position{\href{http://www.mps.mpg.de/sage}{SAGE}, Max-Planck Institute for Solar System Research}
           {Goettingen, Germany}
           {Summer 2016}
           {Research Assistant}
           {
    \textbf{Advisors:}
    Saskia Hekker, George Angelou, Earl Bellinger, Shashi M. Kanbur
    \\
    \textbf{Topic:}
    An asteroseismic analysis of the RGB bump using MESA and ADIPLS
  }

  \position{NASA Jet Propulsion Laboratory}
           {Pasadena, CA, USA}
           {Summer 2015}
           {Summer Intern}
           {
    \textbf{Advisors:}
    Diane Conner and Mark Johnston
    \\
    \textbf{Topic:}
    Development of a web-based tool to assist software engineers at JPL.
  }

  \position{Department of Physics \& Astrophysics, University of Delhi}
           {New Delhi, India}
           {Summer 2014}
           {Research Assistant}
           {
    \textbf{Advisors:}
    Shashi M. Kanbur, H. P. Singh
    \\
    \textbf{Topic:}
    Analysis of several RR Lyrae variable stars in the CSTAR data sets.
  }

\end{rSection}

\begin{rSection}{Awards}{}

  \textbf{NASA Honors Award}, NASA
  \hfill
  \emph{September 28, 2020}\\
  To the InSight Mission Planning and Sequencing Team for developing maintaining and operating a robust Planning and Sequencing System in support of deployment HP3 recovery and science monitoring operations.

  \textbf{NASA Group Achievement Award}, NASA
  \hfill
  \emph{August 28, 2019}\\
  To the InSight Surface Activity Planning Development Team for design and implementation of the new Science Plan Integrator tool suite enabling tactical surface operations.

  \textbf{Successful completion of the Link Complexity and Maintenance Tool}, NASA JPL
  \hfill
  \emph{July 13, 2018}

  \textbf{Development and Delivery of the Link Complexity Scheduling Tool}, NASA JPL
  \hfill
  \emph{Sept. 22, 2017}

\end{rSection}

\begin{rSection}{Technical Skills}{}

\begin{tabular}{ @{} >{\bfseries}l @{\hspace{6ex}} l }
Languages &
Java, Python, C/C++, Javascript, HTML/CSS
\\
Tools & Git, PostgreSQL, Docker, Hasura, Emacs, IntelliJ, Jira
\end{tabular}

\end{rSection}

\end{document}
