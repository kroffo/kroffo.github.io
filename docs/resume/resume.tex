%%%%%%%%%%%%%%%%%%%%%%%%%%%%%%%%%%%%%%%%%
% Medium Length Professional CV
% LaTeX Template
% Version 2.0 (8/5/13)
%
% This template has been downloaded from:
% http://www.LaTeXTemplates.com
%
% Original author:
% Trey Hunner (http://www.treyhunner.com/)
%
% Modifications made by:
% Daniel Wysocki (dwysocki.github.io)
%
% Important note:
% This template requires the resume.cls file to be in the same directory as the
% .tex file. The resume.cls file provides the resume style used for structuring
% the document.
%
%%%%%%%%%%%%%%%%%%%%%%%%%%%%%%%%%%%%%%%%%

\documentclass{resume} % Use the custom resume.cls style

\usepackage[left=0.5in,top=0.1in,right=0.5in,bottom=0.1in]{geometry}
\usepackage{hyperref}
\hypersetup{
  hidelinks,
  colorlinks,
  urlcolor=blue
}
\usepackage{graphicx}

% dingbats fonts
\usepackage{bbding}

\newcommand*\wwwicon{\includegraphics[height=2ex]{globe.eps}}
\newcommand*\mailicon{\includegraphics[height=2ex]{Letter.png}}
\newcommand*\phoneicon{\includegraphics[height=2ex]{phone.jpg}}

\name{Kenny Roffo}
\info{
  \mailicon\hspace{0.10cm}
  \href{mailto:kroffojr@gmail.com}{kroffojr@gmail.com}
  \phoneicon\hspace{0.10cm}(315)-806-7757
  \wwwicon\hspace{0.10cm}
  \href{http://kennyroffo.com}
       {kennyroffo.com}
  \\
  \faGithub\hspace{0.10cm}\href{https://github.com/kroffo/}{\texttt{kroffo}}
}

\begin{document}

\textbf{State University of New York at Oswego}
\\
B.S. Physics, Mathematics, Computer Science, Honors Program, 3.66
\hfill
\emph{May 2017}

\begin{rSection}{Technical Skills}{}

\begin{tabular}{ @{} >{\bfseries}l @{\hspace{6ex}} l }
Languages &
Java, SQL, Python, C/C++, Javascript
\\
Tools &
Git, PostgreSQL, Docker, Jira, Kubernetes, AWS
\end{tabular}

\end{rSection}

\begin{rSection}{Professional Experience}{}
  \position{\href{https://www.fermatadiscovery.com/}{Fermata Discovery Inc.}}
           {New York, NY, USA}
           {Feb 2023 - Present}
           {Software Engineer}
           {
                At this early stage startup I provided my expertise and insight into software development best practices and design principles. I helped the team to understand the importance of thorough testing, high quality code contributions, and constructive code reviews. The backend system was written in Python and deployed through AWS. Neo4J was used for the database, with GraphQL to communicate with the frontend.
           }

  \position{\href{https://percent.com/}{Percent Technologies}}
           {New York, NY, USA}
           {April 2022 - Feb 2023}
           {Backend Software Engineer}
           {
             Percent's system was comprised of a set of Java-based microservices with PostgreSQL for the database. I worked primarily in the Esign area of the system, which included modeling investor profiles in SQL tables,  preparing profile form schemas for the frontend and saving their data to the database on submission, and filling PDF documents programatically. We used Kubernetes and AWS for deployments, and Lens and DBeaver to monitor the system.
           }

  \position{\href{https://www.jpl.nasa.gov/}{NASA Jet Propulsion Laboratory}}
           {Pasadena, CA, USA}
           {June 2017 -- April 2022}
           {Engineering Applications Software Engineer}
           {
        \position{\href{https://github.com/NASA-AMMOS/aerie}{Aerie (Open Source)}}
        {}
        {2019 -- 2022}
        {
          Aerie is a project focused on designing and building a service-based architecture aimed at addressing mission planning needs, including simulation and activity scheduling. I worked as an engineer on the backend which is primarily in Java. Aerie's tech stack included Java, Gradle, Docker, Javalin, Hasura, GraphQL, PostgreSQL and Typescript.
        }

        \position{\href{https://www.jpl.nasa.gov/missions/europa-lander}{Europa Lander}}
        {}
        {2020 -- 2022}
        {
           For Lander I developed a Java-based mission model aimed at simulating different mission scenarios enabling us to explore alternative mission concepts very early on in mission development. I built a highly configurable model that generates and simulates an activity plan for the full mission from landing to to mission end, using JPL's Blackbird simulation engine. This project fostered much growth for me as my first project as the only software engineer, and I learned a lot in the process.
        }

        \position{Flight Software Core (FSWCore)}
        {}
        {2019 -- 2020}
        {
          Contributed to the sequencing engine component of the flight software project FSWCore. This task included design decisions and discussions with one teammate to build a sequencing engine that satisfied requirements while attaining high quality software design. One of my largest responsibilities on this task was to develop the test suite for the sequencing engine, which included nominal success cases, edge case testing, and off-nominal test cases to prove the sequencing engine would not flounder at unexpected scenarios. As part of test development I worked with requirements engineers to improve written requirements where they proved inadequate as written. Work was performed in the C programming language.
        }
        \newpage
        \position{\href{http://www.oswegocountynewsnow.com/news/mars-landing-puts-phoenix-native-suny-oswego-grad-roffo-among/article_968bd574-f34c-11e8-8c5b-b7d557e9f682.html?fbclid=IwAR1iCW3CdQIRr2JIy1f96utXjzzidSJ6vA13jVgvkjKIU6PopaLHYkpbQrw}{InSight}}
        {}
        {2017 -- 2019}
        {
          Worked with two others to develop Python-based tools to perform various tasks from file conversions to generating full web-page reports. Our most notable product was an excel-like UI fully integrated with our simulation and reporting tools. These tools were used to plan and operate the InSight Mars Lander daily once the spacecraft landed on the surface of Mars.
        }

    }
\end{rSection}

\begin{rSection}{Awards}{}
  \textbf{Europa Lander Team Award}, NASA JPL
  \hfill
  \emph{April 1, 2022}

  \textbf{NASA Honors Award}, NASA
  \hfill
  \emph{September 28, 2020}

  \textbf{NASA Group Achievement Award}, NASA
  \hfill
  \emph{August 28, 2019}

  \textbf{Successful completion of the Link Complexity and Maintenance Tool}, NASA JPL
  \hfill
  \emph{July 13, 2018}

  \textbf{Development and Delivery of the Link Complexity Scheduling Tool}, NASA JPL
  \hfill
  \emph{Sept. 22, 2017}

\end{rSection}

\end{document}
