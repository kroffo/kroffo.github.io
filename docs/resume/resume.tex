%%%%%%%%%%%%%%%%%%%%%%%%%%%%%%%%%%%%%%%%%
% Medium Length Professional CV
% LaTeX Template
% Version 2.0 (8/5/13)
%
% This template has been downloaded from:
% http://www.LaTeXTemplates.com
%
% Original author:
% Trey Hunner (http://www.treyhunner.com/)
%
% Modifications made by:
% Daniel Wysocki (dwysocki.github.io)
%
% Important note:
% This template requires the resume.cls file to be in the same directory as the
% .tex file. The resume.cls file provides the resume style used for structuring
% the document.
%
%%%%%%%%%%%%%%%%%%%%%%%%%%%%%%%%%%%%%%%%%

\documentclass{resume} % Use the custom resume.cls style

\usepackage[left=0.5in,top=0.1in,right=0.5in,bottom=0.1in]{geometry}
\usepackage{hyperref}
\hypersetup{
  hidelinks,
  colorlinks,
  urlcolor=blue
}
\usepackage{graphicx}

% dingbats fonts
\usepackage{bbding}

\newcommand*\wwwicon{\includegraphics[height=2ex]{globe.eps}}
\newcommand*\mailicon{\includegraphics[height=2ex]{Letter.png}}
\newcommand*\phoneicon{\includegraphics[height=2ex]{phone.jpg}}

\name{Kenny Roffo}
\info{
  \mailicon\hspace{0.10cm}
  \href{mailto:kroffojr@gmail.com}{kroffojr@gmail.com}
  \phoneicon\hspace{0.10cm}(315)-806-7757
  \wwwicon\hspace{0.10cm}
  \href{http://kennyroffo.com}
       {kennyroffo.com}
  \\
  \faGithub\hspace{0.10cm}\href{https://github.com/kroffo/}{\texttt{kroffo}}
}

\begin{document}

\textbf{State University of New York at Oswego}
\\
B.S. Physics, Mathematics, Computer Science, Honors Program, 3.66
\hfill
\emph{May 2017}

\begin{rSection}{Professional Experience}{}
  \position{Percent Technologies}
           {New York, NY, USA}
           {April 2022 - November 2022}
           {Backend Software Engineer}
           {
             Percent is a Fintech startup with the goal of creating a standard marketplace for private credit transactions. During my time there I worked closely with product managers and other engineers to determine and execute appropriate solutions to problems. My work was primarily done by working with SQL and Java (Springboot), with a heavy emphasis on keeping as much business logic in SQL, as opposed to in the Java code, per the company's engineering decisions.
           }

  \position{NASA Jet Propulsion Laboratory}
           {Pasadena, CA, USA}
           {June 2017 -- April 2022}
           {Engineering Applications Software Engineer}
           {
        \position{\href{https://github.com/NASA-AMMOS/aerie}{Aerie (Open Source)}}
        {}
        {2019 -- 2022}
        {
          On Aerie we have been designing and building a service-based architecture aimed at addressing mission planning needs, including simulation and scheduling needs. I have worked as an engineer on the backend which is primarily in Java. Aerie's tech stack includes Java, Gradle, Docker, Javalin, Hasura, GraphQL, PostgreSQL and Typescript.
        }

        \position{Europa Lander}
        {}
        {2020 -- 2022}
        {
           For Lander I have developed a java-based mission model aimed at simulating different mission scenarios enabling us to explore alternative mission concepts very early on in mission development. I have built a highly configurable model that generates and simulates an activity plan for the full mission from landing to death, using JPL's Blackbird simulation engine.
        }

        \position{Flight Software Core (FSWCore)}
        {}
        {2019 -- 2020}
        {
          Developed tests in C for the sequencing engine component of the flight software project FSWCore. This work involved a bit of requirements engineering, and each test was very tightly tied to a specific requirement being tested.
        }

        \position{\href{http://www.oswegocountynewsnow.com/news/mars-landing-puts-phoenix-native-suny-oswego-grad-roffo-among/article_968bd574-f34c-11e8-8c5b-b7d557e9f682.html?fbclid=IwAR1iCW3CdQIRr2JIy1f96utXjzzidSJ6vA13jVgvkjKIU6PopaLHYkpbQrw}{InSight}}
        {}
        {2017 -- 2019}
        {
          Worked with two others to develop Python-based tools to perform various tasks from file conversions to generating full web-page reports. Our most notable product was an excel-like UI fully integrated with our simulation and reporting tools.
        }

    }
\end{rSection}

\begin{rSection}{Research \& Internship Experience}{}

  \position{NASA Jet Propulsion Laboratory}
           {Pasadena, CA, USA}
           {September 2016 -- December 2016}
           {Software Engineer Intern - Web development}

 \position{\href{http://www.mps.mpg.de/sage}{SAGE}, Max-Planck Institute for Solar System Research}
           {Goettingen, Germany}
           {Summer 2016}
           {Research Assistant - Analysis of Red Giant Branch Bump}

  \position{NASA Jet Propulsion Laboratory}
           {Pasadena, CA, USA}
           {Summer 2015}
           {Software Engineer Intern - Web Development}

  \position{Department of Physics \& Astrophysics, University of Delhi}
           {New Delhi, India}
           {Summer 2014}
           {Research Assistant - Analysis of RR Lyrae variable stars}

\end{rSection}

\begin{rSection}{Technical Skills}{}

\begin{tabular}{ @{} >{\bfseries}l @{\hspace{6ex}} l }
Languages &
Java, Python, C/C++, Javascript, HTML/CSS
\\
Tools &
Git, PostgreSQL, Docker, IntelliJ, Jira
\end{tabular}

\end{rSection}

\begin{rSection}{Awards}{}
  \textbf{Europa Lander Team Award}, NASA JPL
  \hfill
  \emph{April 1, 2022}\\
  Kenny is awarded Europa Lander's 2022 Team Award for esign Sim support and successful execution.

  \textbf{NASA Honors Award}, NASA
  \hfill
  \emph{September 28, 2020}\\
  To the InSight Mission Planning and Sequencing Team for developing maintaining and operating a robust Planning and Sequencing System in support of deployment HP3 recovery and science monitoring operations.

  \textbf{NASA Group Achievement Award}, NASA
  \hfill
  \emph{August 28, 2019}\\
  To the InSight Surface Activity Planning Development Team for design and implementation of the new Science Plan Integrator tool suite enabling tactical surface operations.

  \textbf{Successful completion of the Link Complexity and Maintenance Tool}, NASA JPL
  \hfill
  \emph{July 13, 2018}

  \textbf{Development and Delivery of the Link Complexity Scheduling Tool}, NASA JPL
  \hfill
  \emph{Sept. 22, 2017}

\end{rSection}

\end{document}
